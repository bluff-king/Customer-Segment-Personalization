
\documentclass[t, aspectratio=169]{beamer}
\usepackage[utf8]{vietnam}
\usepackage{amsthm,amsmath,amssymb}
\usepackage{lmodern}
\usepackage{booktabs} % For better looking tables
\usepackage{colortbl} % For colored table cells
\usepackage{graphicx}

% --- CẤU HÌNH LỀ (Phải đặt ở đây) ---
\def\slideContentLeftMargin{0.5cm}  % Lề trái
\def\slideContentRightMargin{1cm}   % Lề phải
\def\hustHeaderHeight{1cm}        % Lề trên (Né thanh tiêu đề)
\def\slideContentBotMargin{3.5cm}   % Lề dưới (Né logo/footer)

% Theme settings
\usepackage[blue, 169]{beamerthemeHUST}

% Customizations
\renewcommand{\titlePageTitleX}{-0.5cm}
\renewcommand{\titlePageTitleY}{-0.5cm}
\renewcommand{\hustTitleAlign}{\alignLeft}
\renewcommand{\hustSubtitleAlign}{\alignLeft}
\renewcommand{\hustAuthorAlign}{\alignLeft}
\renewcommand{\hustInstAlign}{\alignLeft}

% Metadata
\title{CUSTOMER SEGMENTATION \& PERSONALIZATION}
\subtitle{BUSINESS DATA ANALYTICS}
\author{\texorpdfstring{Group 18 \\ Doan Anh Vu - 20225465 \\ Vu Trung Thanh - 20220066 \\ Luong Huu Thanh - 20225458 \\ Nguyen Mau Trung - 20225534}{Group 18}}
\institute{HANOI UNIVERSITY OF SCIENCE AND TECHNOLOGY \\ SCHOOL OF INFORMATION AND COMMUNICATION TECHNOLOGY}
\date{November 2025}

\begin{document}

    \hustintropage{}
    \hustsectionpage{}

    % Title Page
    \husttitlepage

    % Outline
    \begin{frame}[allowframebreaks]{Outline}
        \tableofcontents
    \end{frame}

    \resetPageCount
    \showPageNumber

    % =================================================================
    % SECTION 1: INTRODUCTION & BUSINESS UNDERSTANDING
    % =================================================================
    \useStyleHustFull
    \useThinBar

    \section{Introduction \& Business Understanding}

    \begin{frame}{Context \& Problem Statement}
        \textbf{Context:}
        \begin{itemize}
            \item E-commerce is increasingly competitive.
            \item "One-size-fits-all" marketing leads to low engagement and high churn.
            \item \textbf{Need:} Transition from uniform strategies to targeted, data-driven personalization.
        \end{itemize}

        \vspace{1em}
        \textbf{Project Goal:}
        \begin{itemize}
            \item Develop a comprehensive Customer Segmentation Framework.
            \item Analyze behavioral patterns (views to transactions).
            \item Identify distinct user cohorts.
            \item Deploy tailored retention strategies using the DOC framework.
        \end{itemize}
    \end{frame}

    \begin{frame}{Business Goals \& Key Deliverables}
        \textbf{Objective:} Architect an end-to-end Customer Segmentation \& Personalization System.

        \vspace{0.5em}
        \textbf{Key Deliverables:}
        \begin{enumerate}
            \item \textbf{Segmentation Model:} K-Means approach using RFM + Behavioral metrics.
            \item \textbf{CLV Prediction:} Probabilistic models (BG/NBD + Gamma-Gamma) to identify high-value users.
            \item \textbf{Action Board:} Strategic mapping of segments to marketing actions.
            \item \textbf{Personalization Dashboard:} Visual interface for stakeholder monitoring.
        \end{enumerate}
    \end{frame}

    \begin{frame}{KPI Tree: Metric Decomposition}
        \textbf{North Star Metric: Gross Merchandise Value (GMV)}
        \[ GMV = \text{Traffic} \times \text{Conversion Rate (CR)} \times \text{ATS} \]

        \vspace{0.5em}
        \textbf{Decomposed Metrics:}
        \begin{itemize}
            \item \textbf{Conversion Rate (CR):}
            \[ CR = \text{View-to-Cart} \times \text{Cart-to-Transaction} \]
            \item \textbf{Average Ticket Size (ATS):} Proxy for Order Value.
            \[ ATS \approx \frac{\text{Total Revenue}}{\text{Total Paying Visitors}} \]
        \end{itemize}
        \textit{*Note: ATS is used because precise Order IDs are missing in the dataset.}
    \end{frame}

    \begin{frame}{The DOC Framework}
        To align technical solutions with business needs:
        
        \begin{columns}[T]
            \begin{column}{0.48\textwidth}
                \begin{block}{Decision}
                    "What is the optimal engagement strategy for each customer segment to maximize retention?"
                \end{block}
                \vspace{0.5em}
                \textbf{Criteria (Success Metrics):}
                \begin{itemize}
                    \item Interpretability
                    \item Actionability
                    \item Stability
                \end{itemize}
            \end{column}
            
            \begin{column}{0.48\textwidth}
                \begin{alertblock}{Options (Levers)}
                    \begin{itemize}
                        \item \textbf{VIP Retention:} Exclusivity, loyalty tiers.
                        \item \textbf{Growth:} Cross-sells to increase frequency.
                        \item \textbf{Reactivation:} Discounts for at-risk users.
                        \item \textbf{Acquisition:} Onboarding for non-transactors.
                    \end{itemize}
                \end{alertblock}
            \end{column}
        \end{columns}
    \end{frame}

    \begin{frame}{Risks \& Mitigation Strategies}
        \begin{itemize}
            \item \textbf{Risk 1: Assumption Validity}
            \begin{itemize} \item ATS proxy might underestimate bundle value. \end{itemize}
            
            \item \textbf{Risk 2: Data Sparsity}
            \begin{itemize} \item Users with 1-2 events creates noise.
            \item \textit{Mitigation:} Exclude from complex CLV; use "Cold Start" recs. \end{itemize}
    
            \item \textbf{Risk 3: Observational Bias}
            \begin{itemize} \item Historical data cannot measure causal impact.
            \item \textit{Mitigation:} Proposed Experimentation Framework (A/B Testing). \end{itemize}
        \end{itemize}
    \end{frame}

    % =================================================================
    % SECTION 2: DATA UNDERSTANDING
    % =================================================================
    \useStyleLogoOnly
    \useThickBar

    \section{Data Understanding}

    \begin{frame}{Dataset Overview}
        \textbf{Source:} RetailRocket E-commerce dataset (May -- Sept 2015).

        \begin{table}
            \centering
            \small
            \begin{tabular}{l l l}
                \toprule
                \textbf{Dataset} & \textbf{Rows} & \textbf{Key Features} \\
                \midrule
                Events & ~2.7M & visitorid, event, itemid, transactionid \\
                Item Properties & ~20M & itemid, property, value \\
                Category Tree & 1,669 & categoryid, parentid \\
                \bottomrule
            \end{tabular}
        \end{table}

        \textbf{Data Constraint:}
        \begin{itemize}
            \item \texttt{transactionid} only present in 0.8\% of events.
            \item Requires deriving proxies for basket analysis.
        \end{itemize}
    \end{frame}

    \begin{frame}{Funnel Analysis: The User Journey}
        \begin{columns}[T]
            \begin{column}{0.5\textwidth}
                \textbf{The "Leaky Bucket":}
                \begin{itemize}
                    \item \textbf{View:} 1.4M users (96.7\% of events).
                    \item \textbf{Add to Cart:} 37k users (2.5\%).
                    \item \textbf{Transaction:} 11k users (0.8\%).
                \end{itemize}
                
                \vspace{1em}
                \textbf{Key Insights:}
                \begin{itemize}
                    \item \textbf{Discovery Friction:} 97.3\% drop-off from View to Cart. Needs engagement strategies.
                    \item \textbf{Closing Friction:} 68.9\% drop-off from Cart to Transaction. Needs "nudge" (flash sales).
                \end{itemize}
            \end{column}
            \begin{column}{0.5\textwidth}
                \begin{figure}
                    \centering
                    \fbox{
                        \begin{minipage}[c][4cm]{0.9\textwidth}
                            \centering \vspace{1.5cm} \textbf{[PLACEHOLDER: Event Type Distribution Chart]}
                        \end{minipage}
                    }
                    \caption{Distribution of User Events}
                \end{figure}
            \end{column}
        \end{columns}
    \end{frame}

    \begin{frame}{Exploratory Data Analysis: Patterns}
        \begin{columns}[T]
            \begin{column}{0.48\textwidth}
                \begin{block}{Long-Tail Distribution}
                    A small "head" of popular items/users drives majority of action.
                    \begin{itemize}
                        \item Most items have 1-10 interactions.
                        \item Requires handling outliers in modeling.
                    \end{itemize}
                \end{block}
            \end{column}
            \begin{column}{0.48\textwidth}
                \begin{block}{Temporal Patterns}
                    \begin{itemize}
                        \item \textbf{Weekly:} Peaks in mid-July.
                        \item \textbf{Daily:} Higher engagement on weekdays (Tuesday peak), drops on weekends.
                    \end{itemize}
                \end{block}
            \end{column}
        \end{columns}
        
        \vspace{0.5em}
         \begin{figure}
            \centering
            \fbox{
                \begin{minipage}[c][2.5cm]{0.8\textwidth}
                    \centering \vspace{1cm} \textbf{[PLACEHOLDER: Time Analysis / Day of Week Chart]}
                \end{minipage}
            }
            \caption{Temporal Analysis}
        \end{figure}
    \end{frame}

    % =================================================================
    % SECTION 3: MODELING - SEGMENTATION
    % =================================================================
    \useStyleHustFull
    \useThinBar

    \section{Modeling: Customer Segmentation}

    \begin{frame}{RFM Framework \& Feature Engineering}
        Transforming raw logs into actionable metrics:
        
        \begin{itemize}
            \item \textbf{Recency (R):} Days since last interaction.
            \item \textbf{Frequency (F):} Total days with purchases.
            \item \textbf{Monetary (M):} Total revenue generated.
        \end{itemize}

        \pause
        \textbf{Engineered Feature: Behavioral Conversion Rate (CR)}
        \[ \text{CR} = \frac{\text{Total Transactions}}{\text{Total Views} + \text{Add-to-Carts} + 1} \]
        \begin{itemize}
            \item Distinguishes "Window Shoppers" (high interaction, low intent) from "Targeted Shoppers".
            \item Laplace smoothing (+1) handles low-data users.
        \end{itemize}
    \end{frame}

    \begin{frame}{Preprocessing \& Clustering Strategy}
        \textbf{1. Skewness Handling:}
        \begin{itemize}
            \item Evaluation showed "Long-Tail" on F and CR.
            \item Applied \textbf{Log-Transformation ($\log1p$)} and \textbf{Standardization} to normalize.
        \end{itemize}

        \vspace{0.5em}
        \textbf{2. Dimensionality:}
        \begin{itemize}
            \item Correlation check showing low correlation among R-F-M-CR.
            \item Decision: \textbf{No PCA} needed; kept original 4D space for interpretability.
        \end{itemize}
    \end{frame}

    \begin{frame}{Determining the Optimal Number of Clusters}
        \begin{columns}[T]
            \begin{column}{0.5\textwidth}
                \textbf{Methodology:}
                \begin{enumerate}
                    \item \textbf{Elbow Method:}
                    \begin{itemize} \item Inertia drop slows at $k=3$ and $k=4$. \end{itemize}
                    \item \textbf{Silhouette Analysis:}
                    \begin{itemize} \item Peak score (0.2666) at $k=4$. \end{itemize}
                \end{enumerate}
                \vspace{1em}
                \textit{Decision: $k=4$ provides the most interpretable business segments.}
            \end{column}
            \begin{column}{0.5\textwidth}
                \begin{figure}
                    \centering
                    \fbox{
                        \begin{minipage}[c][4cm]{0.9\textwidth}
                            \centering \vspace{1.5cm} \textbf{[PLACEHOLDER: Elbow \& Silhouette Plot]}
                        \end{minipage}
                    }
                \end{figure}
            \end{column}
        \end{columns}
    \end{frame}

    \begin{frame}{The 4 Customer Segments}
         \begin{columns}[T]
            \begin{column}{0.6\textwidth}
                \textbf{1. Loyal Customers (The "Whales"):}
                \begin{itemize} \item High Frequency, High Monetary. Profit engine. \end{itemize}
                
                \textbf{2. Promising Users (The "Rising Stars"):}
                \begin{itemize} \item High Conversion Rate (57.8\%), moderate spend. \end{itemize}

                \textbf{3. Hibernating (The "At Risk"):}
                \begin{itemize} \item High Recency (inactive), historical value. Needs reactivation. \end{itemize}

                \textbf{4. Non-Transactors (The "Window Shoppers"):}
                \begin{itemize} \item High views, zero transactions. Needs conversion. \end{itemize}
            \end{column}
            \begin{column}{0.5\textwidth}
                 \begin{figure}
                    \centering
                    \fbox{
                        \begin{minipage}[c][4cm]{0.8\textwidth}
                            \centering \vspace{1.5cm} \textbf{[PLACEHOLDER: 3D Cluster Visualization]}
                        \end{minipage}
                    }
                \end{figure}
            \end{column}
        \end{columns}
    \end{frame}

    \begin{frame}{Segment Profiles: Visual Proof}
        \begin{figure}
            \centering
            \fbox{
                \begin{minipage}[c][5cm]{0.85\textwidth}
                    \centering \vspace{2cm} \textbf{[PLACEHOLDER: RFM Heatmap by Segment]}
                \end{minipage}
            }
            \caption{RFM Profiles Heatmap}
        \end{figure}
    \end{frame}

    % =================================================================
    % SECTION 4: MODELING - CLV
    % =================================================================
    \useStyleLogoOnly
    \useThickBar

    \section{Modeling: Customer Lifetime Value (CLV)}

    \begin{frame}{CLV Modeling Framework}
        Evaluation of future value using probabilistic models:

        \begin{block}{1. BG/NBD Model (Frequency)}
            \begin{itemize}
                \item Predicts expected number of future purchases.
                \item Assumption: "Buy-till-you-die" process.
            \end{itemize}
        \end{block}

        \begin{block}{2. Gamma-Gamma Model (Monetary)}
            \begin{itemize}
                \item Predicts average transaction value.
                \item Applied to repeat customers.
            \end{itemize}
        \end{block}

        \textbf{Result:} Forward-looking CLV estimates to complement historical RFM segments.
    \end{frame}

    \begin{frame}{Strategic Value Map: The "Danger Zone"}
        \begin{columns}[T]
            \begin{column}{0.5\textwidth}
                Analysis of Predicted CLV vs. Probability of Being Alive ($P(\text{Alive})$).
                
                \vspace{1em}
                \textbf{The Insight:}
                \begin{itemize}
                    \item \textbf{High CLV + Low P(Alive):} The \textbf{"Danger Zone"}. valuable customers likely to churn.
                    \item \textbf{Strategy:} Highest ROI targets for retention/win-back.
                \end{itemize}
            \end{column}
            \begin{column}{0.5\textwidth}
                \begin{figure}
                    \centering
                    \fbox{
                        \begin{minipage}[c][4cm]{0.9\textwidth}
                            \centering \vspace{1.5cm} \textbf{[PLACEHOLDER: CLV vs P(Alive) Chart]}
                        \end{minipage}
                    }
                    \caption{Strategic Customer Value Map}
                \end{figure}
            \end{column}
        \end{columns}
    \end{frame}

    % =================================================================
    % SECTION 5: EVALUATION & STRATEGY
    % =================================================================
    \useStyleHustFull
    \useThinBar

    \section{Evaluation \& Execution Strategy}

    \begin{frame}{Detailed Segment Metrics}
        \begin{table}
            \centering
            \footnotesize
            \begin{tabular}{l c c c c}
                \toprule
                \textbf{Segment} & \textbf{Recency} & \textbf{Freq} & \textbf{Monetary} & \textbf{CR} \\
                \midrule
                Hibernating & 101.2 days & 1.17 & \$92k & 20\% \\
                \rowcolor[gray]{0.9} Loyal Customers & 55.9 days & 15.42 & \$1.9M & 19\% \\
                Non-Transactors & 33.3 days & 1.23 & \$137k & 18\% \\
                \rowcolor{green!10} Promising Users & 63.9 days & 1.58 & \$178k & \textbf{57\%} \\
                \bottomrule
            \end{tabular}
        \end{table}
        \begin{itemize}
            \item \textbf{Loyal:} High Freq & Monetary.
            \item \textbf{Promising:} exceptional Conversion Rate.
        \end{itemize}
    \end{frame}

    \begin{frame}{Insight: Loyal vs. Promising Users}
        \begin{columns}[T]
            \begin{column}{0.48\textwidth}
                \begin{block}{Loyal Customers}
                    \begin{itemize}
                        \item \textbf{Metrics:} High Frequency (15x), High Spend.
                        \item \textbf{Behavior:} The "Cash Cows".
                        \item \textbf{Strategy:} Maintenance. Don't annoy them.
                    \end{itemize}
                \end{block}
            \end{column}
            \begin{column}{0.48\textwidth}
                \begin{alertblock}{Promising Users}
                    \begin{itemize}
                        \item \textbf{Metrics:} Moderate Spend, but \textbf{57.8\% Conversion Rate}.
                        \item \textbf{Behavior:} High Intent.
                        \item \textbf{Strategy:} \textbf{Growth}. These are future VIPs if nurtured correctly.
                    \end{itemize}
                \end{alertblock}
            \end{column}
        \end{columns}
    \end{frame}

    \begin{frame}{Execution Roadmap}
        Converting analytics into action phases:

        \begin{description}
            \item[Phase 1: Immediate (0-30 days)] \textbf{"Stop the Bleeding"}
            \begin{itemize}
                \item Action: VIP Retention \& Win-back campaigns.
                \item Timing: Tuesdays (Peak activity).
            \end{itemize}
            
            \item[Phase 2: Short-Term (1-3 months)] \textbf{"Nurture Growth"}
            \begin{itemize}
                \item Action: Recommendation engine for "Promising" users.
                \item Goal: Increase Basket Size.
            \end{itemize}

            \item[Phase 3: Ongoing] \textbf{"Optimize"}
            \begin{itemize}
                \item Action: Quarterly re-clustering to track segment migration.
            \end{itemize}
        \end{description}
    \end{frame}

    \begin{frame}{Action Recommendation Matrix}
        \begin{figure}
            \centering
            \fbox{
                \begin{minipage}[c][5cm]{0.8\textwidth}
                    \centering \vspace{2cm} \textbf{[PLACEHOLDER: Action Recommendations Matrix]}
                \end{minipage}
            }
            \caption{Strategy per Segment}
        \end{figure}
    \end{frame}

    \section{Conclusion}

    \begin{frame}{Conclusion}
        \begin{block}{Summary}
            \begin{itemize}
                \item Analyzed 2.7M events to address "leaky bucket" conversion problem.
                \item Identified 4 actionable segments using K-Means (RFM+CR).
                \item Modeled future value (CLV) to prioritize high-risk, high-reward users.
            \end{itemize}
        \end{block}

        \vspace{1em}
        \textbf{Impact:}
        \begin{itemize}
            \item Shifts from generic mass-marketing to precision targeting.
            \item Optimizes budget by focusing on the "Danger Zone" (High Value, High Churn Risk).
            \item Provides a clear roadmap for automated personalization.
        \end{itemize}
    \end{frame}

    % =================================================================
    % CLOSING
    % =================================================================

    \hustcontactpage{\hfill CONTACT INFORMATION \hfill }{
        \textbf{Group:} 18 \\
        \textbf{Email:} group18@example.com \\
        \vspace{0.5cm}
        \textit{Thank you for listening!}
    }

    \useStyleLogoOnly 
    \setbeamertemplate{bibliography item}[text]
    
    \begin{frame}[allowframebreaks]{References}
        \begin{thebibliography}{99}
            \bibitem{clv_kmeans}
            Sa’diah, D. H., \& Fahrudin, N. F.,
            ``Customer lifetime value (CLV) analysis on customer segmentation using RFM model with k-means clustering,''
            \emph{AIP Conference Proceedings}, 2025.

            \bibitem{baesian_clv}
            Karvanen, J., et al.,
            ``Survey data and Bayesian analysis: a cost-efficient way to estimate customer equity,''
            \emph{Quantitative Marketing and Economics}, 2014.

            \bibitem{fader_rfm}
            Fader, P., Hardie, B., \& Lee, K.,
            ``RFM and CLV: Using iso-value curves for customer base analysis,''
            \emph{Journal of Marketing Research}, 2005.

            \bibitem{pareto_nbd}
            Tran, K. G., Nguyen, V. H., \& Ho, T.,
            ``Customer segmentation analysis and customer lifetime value prediction using Pareto/NBD and Gamma-Gamma model,'' 2021.
        \end{thebibliography}
    \end{frame}

    \hustthankyou

\end{document}
